%%%%%%%%%%%%%%%%%%%%%%%%%%%%%%%%%%%%%%%%%
% Twenty Seconds Resume/CV
% LaTeX Template
% Version 1.0 (14/7/16)
%
% Original author:
% Carmine Spagnuolo (cspagnuolo@unisa.it) with major modifications by 
% Vel (vel@LaTeXTemplates.com) and Harsh (harsh.gadgil@gmail.com)
%
% License:
% The MIT License (see included LICENSE file)
%
%%%%%%%%%%%%%%%%%%%%%%%%%%%%%%%%%%%%%%%%%

%----------------------------------------------------------------------------------------
%	PACKAGES AND OTHER DOCUMENT CONFIGURATIONS
%----------------------------------------------------------------------------------------

\documentclass[letterpaper]{twentysecondcv} % a4paper for A4

% Command for printing skill overview bubbles
\newcommand\skills{ 
~
	\smartdiagram[bubble diagram]{
        \textbf{Développeur}\\\textbf{mobile Android},
        \textbf{~~Gradle~~},
        \textbf{~Injection~}\\\textbf{~de dépendance~},
        \textbf{~~Kotlin~~},
        \textbf{~~Flutter~~},
        \textbf{~~~GIT~~~}\\\textbf{~~~SVN~~~},
        \textbf{~~~~~~IOT/BLE~~~~~~},
        \textbf{RXJava}\\\textbf{~~RXAndroid~~}
    }
}

% Programming skill bars
\programming{{C++ $\textbullet$ QT Quick $\textbullet$ Objective C  / 3.5}, {Java $\textbullet$ C\# $\textbullet$ Kotlin / 5}}

% Projects text
\projects{
\textbf{Projets YesItIs} -Applications Monamove (\textbf{BLE}), CoverGarden, Sonde température, Revive Music\\ \\
\textbf{Projet DDI} - Projet IOT - Application comm \textbf{BLE} Android IdealDriver\\ \\
\textbf{Projets Michelin} - Projets confidentiels IOT - Applications en Android et en c\# qui communiquent directement avec des object connectés via \textbf{NFC} ou \textbf{USB}\\ \\
\textbf{CKPay} - Application permettant de créditer des bornes de paiement connectées dans des stations de lavage auto partenaires \\ \\
 \textbf{SMXL} - Application de guide des tailles de vêtements sur Android et iOS
}

%----------------------------------------------------------------------------------------
%	 PERSONAL INFORMATION
%----------------------------------------------------------------------------------------
% If you don't need one or more of the below, just remove the content leaving the command, e.g. \cvnumberphone{}

\cvname{Adrien Boissier} % Your name
\cvjobtitle{ Développeur mobile Android } % Job
% title/career

\cvlinkedin{/in/adrien-boissier-795ab8105}
\cvgithub{trixidis}
\cvnumberphone{06 85 58 44 74} % Phone number
%\cvsite{} % Personal website
\cvmail{boissier38@gmail.com} % Email address

%----------------------------------------------------------------------------------------

\begin{document}

\makeprofile % Print the sidebar

%----------------------------------------------------------------------------------------
%	 ETUDES
%----------------------------------------------------------------------------------------
\section{Etudes}

\begin{twenty} % Environment for a list with descriptions
\\
	\twentyitem
    	{2015 - 2016}
        {}
        {Licence professionnelle option SIL \\ \textnormal{(Plateformes mobiles)}}
        {\href{https://iutweb.u-clermont1.fr/}{IUT d'Aubière}}
        {}
        {}
        
        \\
	\twentyitem
    	{2013 - 2015}
		{}
        {BTS SIO option SLAM \\ \textnormal{(Développement)}}
        {\href{https://www.lyceepeytavin.com/}{Lycée Emile Peytavin, Mende}}
        {}
        {}
	%\twentyitem{<dates>}{<title>}{<organization>}{<location>}{<description>}
\end{twenty}


%----------------------------------------------------------------------------------------
%	 EXPERIENCE
%----------------------------------------------------------------------------------------

\section{Expériences}

\begin{twenty} % Environment for a list with descriptions
\twentyitem
    	{Mars 2019 -}
		{Aujourd'hui}
        {Lead Développeur mobile Android R\&D IOT}
        {\href{https://yesitis.fr/}{YesItIs (startup)}}
        {}
        {\begin{itemize}
        \item Au sein d'une équipe de développeurs mobiles, je développe les applications qui communiquenet avec les différents objets connectés développés en interne et suis en charge des décisions techniques.
         \item  \textbf{Langages : }Android en natif Java et Kotlin, iOS en Swift et Objective C, CrossPlatform avec Flutter et react-native
        \item \textbf{Stack : }RxJava, RxAndroid, RxKotlin, Retrofit, OkHttp, Koin, OrmLite, MosbyMVP, Nfc, RxBle, FlutterBlue, Git
        \item \textbf{Archi : }MVP, MVVM, Bloc
        \end{itemize}}
        \\ \\
\twentyitem
    	{Décembre 2018 -}
		{Mars 2019}
        {Développeur mobile Android R\&D IOT}
        {\href{https://ddi.michelin.com/}{DDI startup Michelin (via Modis)}}
        {}
        {\begin{itemize}
        \item Au sein d'une équipe de développeurs mobiles et embarqués, je développe les outils nécessaires à l'utilisation des devices en bluetooth.
         \item  \textbf{Langages : }Android en natif Kotlin
        \item \textbf{Stack :} RxAndroid, RxKotlin, Retrofit, OkHttp, Realm, MosbyMVP, Kodein, Dagger, RxBle, Git 
        \item \textbf{Archi :} MVP, MVVM
        \end{itemize}}
        \\ \\
\twentyitem
    	{Mai 2017 -}
		{Décembre 2018}
        {Développeur mobile Android R\&D IOT}
        {\href{http://www.michelin.fr/}{Michelin (via Modis)}}
        {}
        {\begin{itemize}
        \item Au sein d'une équipe de 5 développeurs embarqués, je développe les outils nécessaires à la maintenance et au débuggage des objets connectés, grâce à des applications qui communiquent avec le matériel via NFC ou par USB OTG.
        \item Travail en scrum par sprints de 3 semaines.
        \item   \textbf{Langages : }Android en natif Java et Kotlin
        \item \textbf{Stack :} RxJava, RxAndroid, RxKotlin, Retrofit, OkHttp, OrmLite, MosbyMVP, Nfc, USB OTG, Git, SVN
        \item \textbf{Archi :} MVP
        \end{itemize}}
        \\ \\ 
	\twentyitem
    	{Sep 2016 -}
		{Mai 2017}
        {Développeur mobile Android \& iOS IOT}
        {\href{http://www.cksquare.fr/}{CK Square}}
        {}
        {
        {\begin{itemize}
        \item En collaboration avec l'équipe qui développe les systèmes embarqués sur des bornes de paiement dans les stations de lavage auto, ma mission principale fut de développer une application mobile sous Android et iOS permettant grâce à un compte client de créditer directement une borne depuis son téléphone en scannant le QRCode de cette dernière.
        \item \textbf{Langages : }Android en natif Java, iOS en natif Swift
    \end{itemize}}
        }
    \\ \\ 
    \twentyitem
   		{Fev 2016 -}
		{Mai 2016}
        {Développeur mobile Android \& iOS}
        {\href{https://www.smxlapp.com/}{Darbot Mobile}}
        {}
        {
        {\begin{itemize}
        \item Modification de l'application SMXL sur les deux plateformes principales Android et iOS .
        \item  \textbf{Langages : }Android en natif Java, iOS en natif Swift
    \end{itemize}}
        }
   
	%\twentyitem{<dates>}{<title>}{<location>}{<description>}
\end{twenty}




\end{document} 
